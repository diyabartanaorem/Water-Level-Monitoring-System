\section{Summary of the Project}
{
\fontsize{12}{14}\selectfont
\noindent
    The Smart Water Level Monitoring System using a microcontroller successfully achieved its objectives of monitoring and managing water levels in real-time. The system efficiently utilizes ultrasonic sensors to measure the water level, calculates the percentage of tank capacity, and provides clear visual feedback via an OLED display and the Blynk platform. Additionally, the system includes a buzzer alert to notify users when the tank is almost full, preventing water wastage due to overflow. This solution demonstrates the potential of integrating IoT and embedded systems to automate and improve water management in households, industries, and agriculture. The project is a cost-effective, user-friendly, and reliable solution for addressing water monitoring challenges.
} 

\vspace{0.2cm}
\section{Recommendation for Future Research}
\fontsize{12}{14}\selectfont{
\begin{enumerate}
    \item \textbf{Integration with Automated Pump Systems} \\
    Future iterations of the Smart Water Level Monitoring System could incorporate an automated pump control mechanism to enhance its functionality and user convenience. By integrating a relay module or a similar device, the system could automatically start or stop the water pump based on predefined water levels, eliminating the need for manual intervention. For instance, the pump could automatically turn on when the water level falls below a certain threshold and turn off when the tank is almost full. This automation not only reduces the risk of overflow or dry runs but also improves energy efficiency and water resource management. Furthermore, such an enhancement would be especially beneficial for households, industries, and agricultural setups where timely water replenishment is critical.

    \item \textbf{Ultrasonic Waterproof Sensors} \\
    The integration of ultrasonic waterproof sensors, such as the widely used JSN-SR04T, can significantly enhance the accuracy and reliability of the water level monitoring system. These sensors operate by emitting ultrasonic sound waves and measuring the time taken for the waves to return after bouncing off the water's surface. This non-contact method ensures precise measurements without any wear and tear caused by direct exposure to water. Additionally, their waterproof design allows them to function effectively in submerged or high-humidity environments, making them ideal for water tanks, reservoirs, and other applications. The JSN-SR04T, in particular, is known for its durability, affordability, and resistance to harsh environmental conditions, ensuring long-term reliability and low maintenance. Adopting such sensors ensures that the system remains robust and accurate, even under challenging operational conditions.

    \item \textbf{Solar Power Integration} \\
    Integrating solar power into the water level monitoring system offers a sustainable and energy-efficient solution, especially for remote areas with limited or unreliable electricity access. Solar panels can provide the necessary energy to power the sensors, microcontroller, and additional components, making the system self-sufficient and reducing dependency on external power sources. This approach not only lowers operational costs but also aligns with environmental sustainability goals by utilizing renewable energy. In addition, a battery backup can be added to store excess solar energy, ensuring uninterrupted operation during nighttime or cloudy weather. The adoption of solar power not only broadens the applicability of the system but also enhances its resilience in diverse environments, including rural and off-grid areas.

    \item \textbf{Support for Multiple Tanks} \\
    To increase the versatility and scalability of the water level monitoring system, future upgrades could include the ability to monitor multiple tanks simultaneously. This feature would be particularly useful for large-scale applications, such as in residential complexes, industrial plants, or agricultural irrigation systems, where multiple water storage units are used. By incorporating multiple sensor inputs and designing a centralized control interface, users could monitor and manage the water levels of all tanks from a single platform, such as the Blynk app or a web dashboard. Additionally, the system could incorporate separate alerts and automated controls for each tank, further streamlining water management. Such a multi-tank monitoring system would provide a comprehensive solution for efficient water usage and conservation across different sectors.

    \item \textbf{Leakage Detection} \\
    Adding a leakage detection feature to the water level monitoring system would provide an additional layer of functionality and ensure better water conservation. This could be achieved by integrating flow sensors or pressure transducers capable of detecting abnormal flow rates or pressure drops, which are indicative of leaks. When a potential leakage is identified, the system could send real-time alerts to the user via the Blynk platform or trigger an automatic shutoff mechanism to prevent water wastage. Such a feature would be invaluable in both residential and industrial setups, where undetected leaks could lead to significant water loss and increased utility costs. By implementing this upgrade, the system would not only monitor water levels but also actively contribute to minimizing water wastage, aligning with global efforts toward sustainable water resource management.
\end{enumerate}
}